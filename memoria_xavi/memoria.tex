\documentclass[11pt,a4paper,catalan]{article}
\usepackage[utf8]{inputenc}
\usepackage{graphicx} % for images
\usepackage{booktabs} % for tables
\usepackage{fancyhdr} % for header and footers
\usepackage{titlesec}
\usepackage{hyperref} % for links 
\usepackage{minted}
\usepackage{amsmath} % for math annotations
\usepackage{todonotes}

%Definició de noms:
\newcommand{\titleTFG}{Sistema intel·ligent de suport al Tutor d'estudis}
\newcommand{\myname}{Xavier Moreno Liceras}



\setlength{\parindent}{0pt}

%Definició de colors:
\definecolor{bgcode}{rgb}{0.93, 0.93, 0.93}

%\setcounter{secnumdepth}{4}
%\setcounter{tocdepth}{4}
\titleformat{\paragraph}
{\normalfont\normalsize\bfseries}{\theparagraph}{1em}{}
\titlespacing*{\paragraph}
{0pt}{3.25ex plus 1ex minus .2ex}{1.5ex plus .2ex}

\pagestyle{fancy}
\fancyhf{}
\rhead{\titleTFG}
\cfoot{\thepage}
 
\hypersetup{
    unicode=true,          % non-Latin characters in Acrobat’s bookmarks
    pdftitle={\titleTFG},    % title
    pdfsubject={TFG},
    pdfauthor={\myname},     % author
    pdfproducer={\LaTeX},   % producer
    pdfcreator={\myname},   % creator of the document
    pdfkeywords={tutor} {innovació docent} {predicció} {pla d'acció tutorial}, % list of keywords
    colorlinks=true,       % false: boxed links; true: colored links
    linkcolor=blue,          % color of internal links (change box color with linkbordercolor)
    citecolor=blue,        % color of links to bibliography
    urlcolor=blue           % color of external links
}

\begin{document}

\bibstyle{plain}
\thispagestyle{empty}

\begin{titlepage}
\begin{center}
\begin{figure}[h]
\begin{center}
\includegraphics[width=6cm]{img/ub.png}
\end{center}
\end{figure}

\textbf{\LARGE Treball final de grau} \\
\vspace*{.5cm}
\textbf{\LARGE GRAU D'ENGINYERIA INFORMÀTICA } \\
\vspace*{.5cm}
\textbf{\LARGE Facultat de Matemàtiques \\ Universitat de Barcelona} \\
\vspace*{1.5cm}
\rule{\textwidth}{0.1mm}\\
\begin{Huge}
\textbf{\titleTFG} \\
\end{Huge}
\rule{\textwidth}{0.1mm}\\

\vspace{1cm}

\begin{flushright}
\textbf{\LARGE Autor: \myname}

\vspace*{2cm}

\renewcommand{\arraystretch}{1.5}
\begin{tabular}{ll}
\textbf{\Large Director:} & \textbf{\Large Laura Igual } \\
\textbf{\Large Realitzat a:} & \textbf{\Large  Departament   } \\
 & \textbf{\Large Matemàtica Aplicada y Anàlisis} \\
\\
\textbf{\Large Barcelona,} & \textbf{\Large \today }
\end{tabular}

\end{flushright}

\end{center}
\end{titlepage}

\pagenumbering{roman}

\todo{Ha de ser redactat en primer persona del present}
\todo{Revisió de les faltes d'ortografia}

\section*{Abstract}

Goldbach's weak conjecture asserts that every odd integer greater than 5 is the sum of three primes. We study that problem and the proof of it presented by H. A. Helfgott and D. Platt. We focus on the circle method. Finally, we describe a computation that confirms Goldbach's weak conjecture up to $10^{28}$.

\section*{Resum}
Normalment un tutor tutoritza a un conjunt d'alumnes i no dóna temps de mirar detingudament alumne per alumne, això fa que possiblement no es realitzin les accions corresponents per a un alumne. 
%Fins ara un tutor segueix les pautes d'un pla d'acció tutorial (PAT)



\newpage 

\section*{Agraïments}

Vull agrair a ... 

\newpage
\thispagestyle{empty}
{\hypersetup{linkcolor=black}
	\tableofcontents
}
\thispagestyle{empty}
\newpage

\pagenumbering{arabic} 
\setcounter{page}{1}


\section{Introducció}
\newpage
\section{Descripció del problema}
\subsection{Explicar dades} \todo{change title}
% Explicar quines dades tenim, amb quines dades treballem.
\subsection{Ciència de les dades}
La ciència de les dades és el conjunt d'etapes per tal d'arribar a un resultat, en forma de coneixement, a partir d'un conjunt de dades. Aquesta aplica un conjunt de tècniques de diferents àreas, ara com matemàtiques, estadística, teoria de la informació o tecnologia de l'extracció d'informació.
\\
\\
Un projecte de ciència de les dades es separa en diverses etapes:
\begin{enumerate}
	\item \textbf{Preguntes} Què és el que volem explorar? Té sentit el que ens estem plantejant?
	\item \textbf{Adquisició de les dades} Com és la font d'obtenció de les dades? (Base de dades, \textit{Web Scraping})
	\item \textbf{Descripció} Aquesta fase abasta tres processos
	\begin{enumerate}
		\item \textbf{Neteja de dades} Com hem de netejar i separar les dades? (mostres atípiques, filtració, redució de dimensions, normalització, extracció de característiques)
		\item \textbf{Agregació} Com hem de recolectar i resumir les dades? (promig, desviació estàndard, box plots)
		\item \textbf{Enriquiment} Com podem afegir més informació a les nostres dades? (Cerca a altres fonts de dades adicionals)
	\end{enumerate}
	\item \textbf{Descobriment} Podem segmentar les nostres dades per trobar grups naturals i disgregats? (Clusterització, visualització)
	\item \textbf{Anàlisis} Com hem de modelar les nostres dades? (Com estan de relacionades cada variable?, Com podem determinar quines són les variables importants?)
	\item \textbf{Predicció} A partir de les dades que tenim, que podem predir del futur? (Regresions, clasificadors, recomanadors)
	\item \textbf{Evaluació} Com de segur estem dels nostres resultats? (Proves estadístiques, rendiment del model)
	
\end{enumerate}
\subsection{Etapes del projecte}

\subsubsection{Preguntes}
La primera etapa va ser el plantejament de les preguntes que voliem resoldre. A partir de la plataforma trello, entre els participants del projecte vam plantejar preguntes, les quals entre tots decidiem amb quines preguntes ens quedariem i respondriem. Moltes de les preguntes no podíem saber si les podíem respondre fins que ens arribessin les dades, ja que depeniem totalment de la informació que contenien les dades.

\subsubsection{Adquisició}
L'adquisició de les dades va ser a partir del Vicerectorat de Política Docent. Les dades ens van arribar a través d'una fulla de càlcul. Tot i que les dades vinguessin anonimitzades i tractades pel departament corresponent, vam haver de fer una neteja de dades.

\subsubsection{Neteja de dades}
En aquesta etapa he hagut de netejar les dades per tal de poder treballar amb elles. Aquestes van ser les netejes que vaig fer:

\subparagraph{Canvi de format}
Com ja he explicat abans les dades ens van arribar en forma de fulla de càlcul, on en cada fulla havía una taula amb diferent informació. Per poder manipular-les millores des de Python, vaig haver de separar cada fulla en un fitxer amb format \textit{csv}, de tal manera que va quedar un fitxer \textit{csv} per taula.

\subparagraph{Canvi de nom de les columnes}
Per tal de poder creuar les diferents taules, els noms de les columnes havien de ser el mateix.

\subparagraph{Enriquiment de les dades}
A partir d'una font hem pogut adquirir el curs i semestre que es cursa cada assignatura, per tant el que faig és creuar aquestes dades amb les dades que tinc de cada assignatura per tal de tenir més informació per assignatura.

\subparagraph{Unió de graus}
L'any 2009 el grau en Enginyeria Informàtica de la UB tenia com a codi \textit{G1041}, però a partir de l'any 2010 el codi va passar a ser \textit{G1077}. Les assignatures eren les mateixes, tot i que tenien codis diferents també. Vam procedir a fer la unió dels \textit{G1041} amb \textit{G1077}, per tal de no perdre informació rellevant, ni considerar-la per separat.

\subparagraph{Eliminació del curs 2014, segon semestre}
Explorant les dades em vaig adonar que gent que s'havia matrículat aquest any 2014, però encara no havien acabat de cursar l'assignatura, en aquesta els hi apareixia un 0. Això feia que dintre de les notes dels alumnes haguéssin dades incoherents, per aquesta raó vam decidir eliminar totes les notes del segon semestre i de l'any 2014  que són un 10.91\% de les notes.

\todo{Elimino o no al final?}

\subparagraph{Normalitació de les notes}
Per tal d'evitar els canvis de mitja i variança en cada assignatura cursada per any, ja sigui per un canvi de professor, canvi de pla docent, diferents promocions, \ldots vam decidir normalitzar les notes per any i per assignatura aplicat una normalització d'unitat tipificada en la qual s'aplica per cada dada la següent fórmula:
$$ z = \frac{x - \mu}{\sigma}, $$

on $\mu$ és el promig per any i per assignatura, i $\sigma$ és la desviació estàndard per any i per assignatura. Amb això aconseguim mitja 0 i desviació estàndard 1.

\subsubsection{Clusterització}
Aquesta etapa era necessaria per poder respondre a una des les preguntes plantejades, i és: \textit{Hi ha diferents perfils d'alumnes?}. Per tal de  respondre a aquesta pregunta he aplicat mètodes de clusterització a partir de les notes dels alumnes diferenciats per cursos.

\subsubsection{Predicció}
La predicció, com s'ha explicat abans, és la predicció del futur a partir de les dades disponibles. En aquest cas hem volgut predir les notes que pot arribar a treure un alumne en base a les notes que ha tret en cursos anteriors.

\subsubsection{Evaluació}
Un cop construida la predicció, hem d'avaluar quant de bona és. Per això vaig agafar un 10\% de les meves dades per poder testejar i comprovar la tassa d'encerts de la predicció.

\newpage

\subsection{Preguntes plantejades} \todo{change title}
% Hi ha diferents perfils d'alumnes?
% Quina es la tassa d'abandonament per cada tipus de perfil?
% Cadascun d'aquests perfils amb quin perfil de provinença encaixa?
% Predicció de notes per fer un ranking de dificultats
\paragraph{Hi ha diferents perfils d'alumnes?}
A partir de la distribució de les notes de cada alumne per cada assignatura que ha fet, podem determinar que hi ha diferents perfils d'estudiants? Això és el que ens estem preguntant. He agafat tots els alumnes que hagin cursat totes les assignatures de primer i després amb segon, tant al grau d'Enginyeria Informàtica com al grau de Matemàtiques. La experiència ens diu que hi han alumnes bons en programació i dolents en les assignatures de matemàtiques a primer del grau d'Enginyeria Informàtica. Però per a la resta de cursos, quins perfil podem trobar? Ara que tenim les dades això ho podem saber, convertirem les dades en coneixement.

\paragraph{Quina es la tassa d'abandonament per cada tipus de perfil?}
A partir dels perfils que han sigut determinats en la pregunta anterior, quin és el percentatge d'abandonament per cadascun d'aquests. Volem saber si és cert que els alumnes cauen al perfil d'alumnes que ho suspenen tot són els que solen abandonar la carrera. Fins ara això sembla força obvi, però ho podem demostrar amb dades i corroborar-ho.

\paragraph{Cadascun d'aquests perfils amb quin perfil de provinença encaixa?}
Al llarg dels anys s'ha pogut notar que els alumnes que venen d'un Cicle Formatiu de Grau Superior (CFGS) solen ser alumnes que els hi dóna malament les assignatures relacionades amb les matemàtiquies i són força bons en programació. Ara bé, això és cert? Per això ens plantegem aquesta pregunta, a partir de perfils d'origen, volem saber amb quin cluster destí van a parar. En aquest cas hem fet els següents creuaments per cada grau:

\begin{table}[h]
\centering
\begin{tabular}{@{}cc@{}}
\toprule
{\bf Origen}               & {\bf Destí}                \\ \midrule
Via d'accés                & Perfil d'alumnes de primer \\
Perfil d'alumnes de primer & Perfil d'alumnes de segon  \\ \bottomrule
\end{tabular}
\end{table}
Els perfils d'alumnes de primer i segon són els perfils determinats a la primera pregunta, i els perfils de via d'accés que hem seleccionat han sigut els següents:
\begin{enumerate}
	\item Batxillerat
	\item Cicle Formatiu de Grau Superior
	\item Diplomats, Licenciats
	\item Salt d'Universitat
\end{enumerate}

\paragraph{Predicció de notes per fer un ranking de dificultats}
A partir de les notes que ha tret un alumne en el seu passat, podem predir quines assignatures li aniran bé i malament en el futur? Bé, això és el que ens plantejem en aquesta última pregunta, volem recomanar a un alumne en quines assignatures anirà fluix per així pogui reforçar més el temari que es donarà en aquella assignatura. Recordem que la finalitat d'aquest projecte es que aquesta eina sigui un suport per al tutor, és a dir, la predicció no ens dirà el que ha de fer un alumne, aquesta decisió es delega al tutor que a partir d'aquesta eina en decidirà que fer.
\newpage

\section{Planificació}
\subsection{Tasques}
% plantejament de preguntes
% procés de data cleaning
% clusterització
% predicció
% evaluació
\subsection{Diagrama de Gantt}
\subsection{Evaluació económica}
\newpage

\section{Desenvolupament del projecte}
\subsection{Eines}
\subsubsection{Eines de suport}
Aquestes són les eines de suport que m'han ajudat al llarg del treball per tal de fer més còmode la seva organització tant personal com per equip.

\paragraph{GitHub}
GitHub és una plataforma online per desenvolupar projectes software de forma col·laborativa. Aquesta plataforma utilitza un control de versions anomenat Git. La finalitat de GitHub és l'emmagatzenament massiu de projectes amb codi font obert. Per això hem optat per la utilització de GitHub, ja que que volem que el nostre codi el pogui veure tothom i que qualsevol que el necessiti per fer la seva investigació, el pogui utilitzar.

\paragraph{Bitbucket}
Bitbucket és una plataforma semblant a GitHub, però amb el servei d'un altre control de versions com Mercurial a més de Git. Bitbucket té l'advantatge de permetre crear repositoris privats de forma gratuïta. Aquesta plataforma va bé per a l'inici d'un projecte on es fan molts canvis en el codi, ja que pots tenir el codi en privat, i un cop el codi ja agafa forma es pot migrar a GitHub. Això és el que hem fet nosaltres en el projecte, començar amb Bitbucket i després passar-nos a GitHub amb el codi font obert.

\paragraph{Trello}
Per últim com eina de suport, hem fet servir Trello, una plataforma online que permet una comunicació més clara entre els membres d'un projecte. Amb Trello pots crear projectes i cada projecte conté un conjunt de llistes que s'omplen de tasques. Nosaltres hem fet servir Trello, per comunicar-nos amb la tutora i tenir present una planificació per tal d'organitzar-nos millor.

\newpage
\subsubsection{Eines de programació}
En aquesta secció trobarem amb el llenguatge de programació i conjunt de llibreries que hem treballat.

\paragraph{Python}
Python és un llenguatge d'alt nivell interpretat. Remarquen molt la fàcil lectura del seus codis, per això té una sintaxis molt semblant a un pseudocodi. Python és un llenguatge de codi obert i desenvolupat per \textit{Python Software Foundation}, una organització sense ànim de lucre. Vam escollir Python en el seu moment per dues simples raons; per ser un llenguatge de scripting i per la seves llibreries relacionades amb el tractament de dades (com \hyperlink{pandas}{Pandas}, \hyperlink{numpy}{NumPy} o \hyperlink{sklearn}{Scikit-learn}).


\hypertarget{pandas}{
	\paragraph{Pandas}
}
Pandas és una biblioteca informàtica escrita en python per a la manipulació i anàlisi de dades. Especialment va bé per al tractament de taules alhora de fer consultes, o per a l'agrupació i agrecació d'informació.

\hypertarget{numpy}{
	\paragraph{NumPy}
}
Numpy és una biblioteca informàtica de Python per operar amb vectors i matrius d'una forma més extensa a la que et permet el mateix llenguatge Python, la qual conté tot un conjunt de funcions matemàtiques d'alt nivell per treballar amb aquests vectors i matrius.

\hypertarget{sklearn}{
	\paragraph{Scikit-learn}
}
Scikit-learn és una biblioteca informàtica orientada a l'aprenentatge automàtic per a Python. Té suport per classificadors, regressors i clustering. Per aquest projecte hem fet servir clustering i regressors. En la secció de \hyperlink{tecniquesutilitzades}{Tècniques utilitzades} es detalla cada tècnica utilitzada d'aquesta biblioteca informàtica.

\paragraph{Bokeh}
Bokeh és una biblioteca informàtica per a la visualització interactiva de dades dirigida als navegadors per a la seva presentació a través d'HTML i JavaScript. Bokeh té el suport per a gràfiques específiques com diagrames de barra, box plots o time series, però a banda d'aquests gràfics pots dibuixar sobre un gràfic amb elements bàsics com cercles, línies, rectangles, entre altres.

\paragraph{Seaborn}
Per últim tenim Seaborn que també és una biblioteca informàtica per a visualitazció de dades com Bokeh, amb gràfiques molt més específiques. A més té una part de la biblioteca informàtica dedicada a les paletes de colors i la qual permet escollir un conjunt de colors afavorits per mostrar les dades.

\begin{figure}[h]
\begin{minted}
[
framesep=5mm,
baselinestretch=1.2,
bgcolor=bgcode,
fontsize=\footnotesize
]
{python}
%matplotlib inline
import seaborn as sbn
palette = sbn.color_palette("hls", 5)
sbn.palplot(palette)
\end{minted}
\begin{center}
\includegraphics[width=8cm]{img/palplot_seaborn.png}
\end{center}
\caption{Elecció d'una paleta de 5 colors}
\end{figure}


\subsubsection{Eines d'edició}
\paragraph{IPython notebook}
Ipython notebook és un editor per a l'entorn de Python. La filosofia \textit{notebook} s'emprea per tenir un codi molt més llegible i a més tenir explicacions d'allò que es programa, ja que es pot barrejar codi, la sortida del codi, markdown, HTML, entre altres. Hem optat per escollit aquest entorn d'edició ja que en un projecte de ciència de les dades s'han de veure resutats constants i poder-los comentar.

\paragraph{Texmaker}
Aquesta és una eina d'edició de \LaTeX, la qual permet poder generar informes, documents, llibres d'una forma més programàtica. A partir d'un etiquetatge estipulat podem generar documents amb un estil predefinit com el d'aquesta memòria.

\newpage
\hypertarget{tecniquesutilitzades}{
	\subsection{Tècniques utilitzades}
}
\subsubsection{Clusterització (Agrupacions)}
La clusterització és molt important en el món de les dades si el que volem es reconèixer diferents grups de ítems de les nostres dades, en el nostre cas d'alumnes. Per això, abans de veure els resultats i experiments explorats, cal entendre les diferències entre les diferents tècniques de clusterització. En aquest projecte hem fet servir dues tècniques, on l'objectiu d'elles és el mateix, desframentar les dades i trobar diferents grups d'alumnes. Aquestes dues tècniquies són K-means i MeanShift, ambdues implementades en la biblioteca informàtica de Scikit-learn. \todo{link}

\paragraph{K-means}
\textit{K-means} probablement és un dels algoritmes d'agrupació més conegut. Partint de $n$ elements, segmenta aquests $n$ elements en $k$ grups (entrada obligatòria de l'algoritme) on cada element pertany al grup més proper a la mitjana. L'algoritme de \textit{K-means} està descrit per la següent fórmula:
\\
\\
Tenint un conjunt d'elements $(x_1, x_2, \ldots, x_n)$ on cada elements és un vector $d$ dimensional, \textit{K-means} construeix una partició dels elements en $k$ grups, on $k <= n$ quedant $S = \{S_1, S_2, \ldots, S_k\}$. Amb la finalitat de minimitzar la suma dels quadrats dintre de cada grup:

$$ \underset{S} {arg\,min} \sum_{i=1}^{k} \sum_{x_j \in S_i} \left\| x_j - \mu_i \right\|^2 $$

on $\mu_i$ és el centroide dels punts del conjunt $S_i$, és a dir, el punt mig.
\\
\\
Com es veu en la fòrmula, aquest algoritme depén d'una $k$, per determinar agrupacions, per tant \textit{K-means} ha de rebre com paràmetre d'entrada quants grups busquem. També podem pensar que depén del centroide $\mu_i$, però no es necessari, ja que aquest convergeix si apliquem iteracions sobre aquesta fòrmula.

\begin{figure}[h]
\centering
\missingfigure{\Huge Nice this triangle?}
\caption{Some caption}
\label{nicefigure}
\end{figure}

\paragraph{MeanShift}
\paragraph{Mètriques utilitzades}
\newpage
\subsubsection{Predicció}
\paragraph{Recomanador}
\paragraph{Random Forest Regressor}
\paragraph{Regressor lineal}
\paragraph{Mètriques utilitzades}
\subsubsection{Reducció de dimensions}
\paragraph{PCA}
\newpage

\section{Experiments i resultats}

\begin{table}[h]
\centering
\begin{tabular}{@{}llll@{}}
\toprule
\textit{\textbf{Algoritme$\setminus$Mètriques}}   & \textbf{MAE} & \textbf{MSE} & \textbf{PCC} \\ \midrule
\textbf{Recomanador col·laboratiu}      & 1.231        & 2.997        & 0.335        \\
\textbf{Recomanador basat en contingut} & 1.197        & 2.905        & 0.403        \\
\textbf{Random Forest Regressor}        & 1.134        & 2.584        & 0.490        \\
\textbf{Linear Regressor}               & 1.175        & 2.720        & 0.462        \\ \bottomrule
\end{tabular}
\caption{Dades no normalitzades}
\end{table}

\begin{table}[h]
\centering
\begin{tabular}{@{}llll@{}}
\toprule
\textit{\textbf{Mètriques/Algoritme}}   & \textbf{MAE} & \textbf{MSE} & \textbf{PCC} \\ \midrule
\textbf{Recomanador col·laboratiu}      & 0.558        & 0.669        & 0.069        \\
\textbf{Recomanador basat en contingut} & 0.531        & 0.660        & 0.358        \\
\textbf{Random Forest Regressor}        & 0.509        & 0.565        & 0.393        \\
\textbf{Linear Regressor}               & 0.538        & 0.648        & 0.462        \\ \bottomrule
\end{tabular}
\caption{Dades normalitzades}
\end{table}

% Preguntes:
%   - Hi ha perfils d'estudiants basats en les notes?
%   - Un dels clusters correspon als alumnes que abandonen?
%   - Quin perfil d'entrada té cada cluster?
%   - Ranking de dificultat de les assignatures que ha de fer (que ha matriculat) un alumne
\newpage
\section{Conclusions i treball futur}
\newpage
\section{Bibliografia}
\end{document}